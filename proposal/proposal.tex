\documentclass{article}

\usepackage{amsmath,amsfonts,amsthm,amssymb}
\usepackage{setspace}
\usepackage{fancyhdr}
\usepackage{lastpage}
\usepackage{chngpage}
\usepackage{hyperref}
\usepackage[usenames,dvipsnames]{color}
\usepackage{graphicx,float,wrapfig}
\usepackage{inconsolata}
\usepackage{natbib}

% In case you need to adjust margins:
\topmargin=-0.45in
\evensidemargin=0in
\oddsidemargin=0in
\textwidth=6.5in
\textheight=9.0in
\headsep=0.25in

%\setlength{\parindent}{0pt}
\setlength{\parskip}{2ex}

% Setup the header and footer
\pagestyle{fancy}
\lhead{Trevor Bekolay, Comprehensive II}
\rhead{\thepage\ of\ \protect\pageref{LastPage}}
\lfoot{}
\cfoot{}
\rfoot{}
\renewcommand\headrulewidth{0.4pt}
\renewcommand\footrulewidth{0pt}

% Make title
\title{Comp-II: Insert title here}
\date{\today}
\author{\textbf{Trevor Bekolay}}
%%%%%%%%%%%%%%%%%%%%%%%%%%%%%%%%%%%%%%%%%%%%%%%%%%%%%%%%%%%%%
\begin{document}

\maketitle

\begin{abstract}
  Abstract
\end{abstract}

\section{Introduction}

\section{Speech recognition}

\subsection{Auditory periphery modelling}

\subsection{Open research questions}

Can using auditory periphery signals
instead of waveforms
result in better speech recognition?

We'll try to do a small subset of speech recognition.

\section{Speech synthesis}

\subsection{Articulatory speech synthesis}

\subsection{Open research questions}

Can current vocal tract models give human-quality speech?

We'll try using modern motor control to generate target sounds.
We can use mngu0 sounds and fMRI for generating initial trajectories

\section{Closed loop modelling}

\subsection{Open research questions}

Can synthesized speech
influence auditory sensory learning,
and vice versa?

We'll try to hook them up together
and learn via babbling and outside teachers.

\section{Spiking neural networks}

\subsection{Open research questions}

Can the previously discussed items
be done in a biologically plausible manner?
Are there any practical benefits to doing this?

We'll use the NEF to model some aspects of this.
Some things,
like representing the articular parameters with neurons,
are trivial but provide good practical benefits
since computer speech sound mechanical due to no noise.

\subsection{Neural Engineering Framework}

\subsection{Learning}

\end{document}
